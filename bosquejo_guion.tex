\documentclass[12pt, a4paper]{article}
\usepackage{mathptmx}
\usepackage[margin = 1in]{geometry}
\usepackage[spanish]{babel}

\usepackage{ragged2e}
\justifying

\title{Pirulí}
\author{Equipo}

\begin{document}
  \maketitle

  %%% This doesn't work
  %%%\pagebreak

  \section*{Descripción del personaje principal.}

  Hay que ponerle vida, pasado, futuro, ¿por qué se comporta así este personaje?


  Debe durar de 3 a 8 minutos.
  La portada hacen los japoneses, después, después de grabar todo. Carpeta creativa, SIN PURPURINA, ALGO QUE NO ENSUCIE, CON CARÁTULA,
  siguiente hoja, integrantes con funciones. Los personajes hablamos como nosotros, como paraguayo, no debe ser surrealista. 

  EL GUION DEBE ESTAR EN ORDEN. TODO se filma de corrido. 

  Número de escena, ambiente, descripción del ambiente y el guion, lo que hace cada personaje. Depende el nivel del audio, es lo más difícil, pero a la vez lo más 
  importante. 

  Música no clásica europea, música. No debe ser Europeo, hay que salvaguardar nuestra cultura. Algún rock paraguayo, alguna canción en español, 
  en guaraní; puede ser clásica paraguaya. 

  \vspace{7pt}
  LO MÁS IMPORTANTE: Pirulí es un niño muy cabezudo, tan cabezudo que fue condenado a un final trájico. Su madre no lo aguantaba, el centro de su vida
  eran las macanadas, no podía estar tres días seguidos sin hacer pasar malos ratos a su madre. ¿Por qué su madre Catarina siempre?,
  porque era ella quien lo protegía de su padre, un borracho panzón que trabaja a tiempo completo y que solo vuelve para suplirse de 
  alcohol y reprochar a sus hijos y esposa.

  No hay que divagarse como si estuviéramos en Hollywood, debe ser algo realizable.


  \hfill\begin{minipage}{0.55\linewidth}
  ``Sé que no hay tiempo, pero igual no hay tiempo''

  ---Profe de literatura, Susana Alvarenga
  \end{minipage}

  \section{EN LA CALLE, EN FRENTE DE UNA CASA - NOCHE}
  Un plano que recorre de costado una pared de ladrillos, cambia gradual pero a la vez de manera rápida, al son de una luz que surge del fósforo
  que predió Pirulí para detonar una bomba, así se ilumina su rostro de cabezudo y se enfoca al medidor de la ANDE;
  prende la mecha, la mete dentro del medidor, cierra la puertita de chapa y sale corriendo al desenfoque y del posterior estallido del cebollón.

  Prende, sale corriendo, se enfoca a la casa y la pilastra pero no de frente, de tal manera que no se vea directamente la explosión, 
  pero sí la iluminación y el sonido reales de la misma.

  De esta manera la casa se queda sin luz, con su respectivo plano.   %Va Pirulí caminando, está en medio de una macanada.

  \vspace{7pt}
  Transición imaginaria.
  \vspace{70pt}

  \textbf{Aún no sé dónde grabar ésta escena:}
  \vspace{7pt}

  Pirulí recibe múltiples mensajes de su madre preguntándole dónde está, inclusive una llamada, (aquí se puede mostrar su preocupación en un plano
  a parte, de ella llamando y quejándose con ``Ahhhh che dio dame fuerza, mba'ejeýmapa ojapo che memby, umia ha'e hína'') pero decide 
  ignorar todo para contarle a su socio que ya hizo volar la pilastra del señor que lo recriminó en el San Juan por haber sido muy grosero 
  con una señorita.
  % pegó en el San Juan, unas semanas atrás

  \section{PATIO CON MUCHO VERDE - DE SIESTA}
  LA HONDITA: Hay un pajarito, dispara un bodoque, y rompe un ventana, o algo más.

  \vspace{7pt}
  \textbf{Cómo es el lugar:}

  El jerutí se encuentra en lugares verdes, y semi-abiertos, rodeado de árboles a los que escapar en todo momento, es la presa de Pirulí.
  Ya que busca un Jerutí, hay que conseguir grabar el canto de uno: \textit{uuuu-uuuu}.

  Se enfoca la hondita en su mano mientras camina, se visualiza una bolsita a su costado izquierdo que contiene bodoques (tengo
  esa utilería).

  Está buscando aves para asesinar a tiro limpio. Agachado y sigiloso como un Ninja. Mira ahí, mira allá, es brusco y bruto,
  apunta a cualquier cosa que se mueve.

  Pirulí escucha el canto, se pone en alerta, ve un Jeruti, apunta, dispara, y rompe el vidrio.

  Pirulí escucha el canto, se pone en alerta, ve un Jeruti que sale volando, dispara y le acierta a alguien en la espalda.

  NOTA: BRAY TIENE VIDRIOS PARA ROMPER. PERFECTO PARA LA ESCENA.

  \section{Casa de Pirulí, por la siesta, posiblemente de noche, ya que debemos encontrar un sapo}
  Se introduce un plano relleno.
  \vspace{7pt}

  Como necesitamos un sapo, que salen en días de lluvia, quedaría bien como escena final, ya que la lluvia puede presagiar el final del corto,
  que culmina con su madre desesperada sin saber qué hacer. El padre estaba de parranda aquel sábado.
  \vspace{7pt}

  Su madre está durmiendo cómodamente hasta que Pirulí decide hacerle una broma. A ella le da pánico los sapos. Pirulí atrapa uno en el patio
  y lo pone a dormir junto con su madre, quien se enoja tanto después del infartante susto, que pega un golpe certero a Pirulí, poniendo 
  fin a su carrera de macanadas.
  \vspace{7pt}


  Me imagino que puede empezar con planos de la lluvia al raz del pasto, plagado de sapos saltando, iluminado de manera que no se note
  artificial, en caso de ser de noche; si llueve de día, mejor. Hay que filmar cómo Pirulí atrapa un sapo en la lluvia, por ende él
  no debe tener miedo de agarrarlos y tampoco de contraer un resfriado. La iluminación puede provenir de un foco de alumbrado público al fondo,
  o de un foco nuestro que haga ese papel, pero no tenemos focos ni menos que sean impermeables.
    % Pirulí y sus amigos encuentran un sapo, lo ponen en alguna parte, para asustar a su socio, que es sapofóbico; lo ponen en su pecho, despierta y...

  \section{Posible escena}
  Por el momento pienso que tres macanadas ya son suficientes, saturar de tantas puede ser frustrante, quiero que el corto sea lo más conciso
  y breve posible, de \textbf{5 minutos a lo sumo}.

  Agarra el celular de su mamá para instalar jueguitos, instala un virus y deja sin datos importantes a su madre.
  El celular es un Samsung Galaxy J2 Prime del siglo XV antes de Cristo, que tengo disponible para terminar de fundir.

  \section*{Notas}
  Hay que incluir a la actualidad, la tecnología de alguna manera. Que su madre le manda mesajes de texto o algo por el estilo. Pienso que 
  puedo hacerlo al final o al inicio de la primera escena, ya que es tarde y él no llega a casa, tal vez se escucha ``Piruliiiiii, Piruliiiiii''
  al fondo; ahora me imagino que puede quedar bien después de la explosión, ya que eso justificaría aún más la preocupación de su madre 
  para llamar a Pirulí.

  Pienso en qué come Pirulí, dónde duerme, cómo es su vida con su madre, ¿cómo consigue escaparse?, ya que en ningún caso tiene permiso.

  Aún queda discutir si en la calle debe haber o no gente, si exactamente a qué hora debe ser, si estará con su socio o no, podemos inventar
  uno aunque es trabajo extra.

  \section*{Tres posibles escenarios:}
  \itemize{
    \item La escuela.
    \item La calle.

    % Parlante: un grupo de tipos que estaban en Eusebio ayala a FULLLLLL volumen, con el objetivo de ensordecer. Pasó una moto al lado que se enojó muchísimo.
    % El parlante era rojo, casi se rompen las ventanas.

    \item La casa.
  }
  % Pone algo en la olla, algo relacionado con el fuego, pero ya hicimos algo relacionado al fuego, no podemos repetirlo.

  \end{document}
